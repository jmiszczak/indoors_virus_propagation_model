\documentclass[11pt,a4paper]{article}

\usepackage[utf8]{inputenc}
\usepackage[OT4]{fontenc}
\usepackage{amsmath}
\usepackage{amssymb}
\usepackage{graphicx}
\usepackage{fullpage}

\title{Agent-based indoors virus propagation model}
\author{Jarosław Miszczak}

\begin{document}
\maketitle

%%%%%%%%%%%%%%%%%%%%%%%%%%%%%%%%%%%%%%%%%%%%%%%%%%%%%%%%%%%%%%%%%%%%%%%%%%%%%%%%
\section{Introduction}
%%%%%%%%%%%%%%%%%%%%%%%%%%%%%%%%%%%%%%%%%%%%%%%%%%%%%%%%%%%%%%%%%%%%%%%%%%%%%%%%


%%%%%%%%%%%%%%%%%%%%%%%%%%%%%%%%%%%%%%%%%%%%%%%%%%%%%%%%%%%%%%%%%%%%%%%%%%%%%%%%
\section{Agent-based model}
%%%%%%%%%%%%%%%%%%%%%%%%%%%%%%%%%%%%%%%%%%%%%%%%%%%%%%%%%%%%%%%%%%%%%%%%%%%%%%%%

\begin{figure}[ht!]
\includegraphics[width=\textwidth]{plots/model-gui.png}
\caption{User interface for the NetLogo model of indoors virus propagation with generated configuration.}
\label{fig:gui-world-3}
\end{figure}


%%%%%%%%%%%%%%%%%%%%%%%%%%%%%%%%%%%%%%%%%%%%%%%%%%%%%%%%%%%%%%%%%%%%%%%%%%%%%%%%
\section{Simulation results}
%%%%%%%%%%%%%%%%%%%%%%%%%%%%%%%%%%%%%%%%%%%%%%%%%%%%%%%%%%%%%%%%%%%%%%%%%%%%%%%%

\begin{figure}[ht!]
\includegraphics{plots/sick_increase_large_s_pop100.pdf}
\caption{Mean increase in the number of infected agents for different values of mobility parameter $\mu$. Each plot represents the absolute increase of the number of infected agents in the population of 100 agents with different number of initially infected agents, ranging from $1$ to $10$,  and with different values of probability contaminating the items.}
\label{fig:sick_increase_large_s_pop100}
\end{figure}


\begin{figure}[ht!]
\includegraphics{plots/sick_increase_large_s_pop50.pdf}
\caption{Mean increase in the number of infected agents for different values of mobility parameter $\mu$ for the population of 50 agents. Other parameters are described in caption of Fig.~\ref{fig:sick_increase_large_s_pop100} and the colour scale is the same. One should note that even for the significant portion of the infected agents in the population, the spread of the disease is limited.}
\label{fig:sick_increase_large_s_pop50}
\end{figure}



\begin{figure}[ht!]
\includegraphics{plots/sick_increase_small_s_pop100_world-3.pdf}
\caption{Mean increase in the number of infected agents for different values of mobility parameter $\mu$ with the contamination parameters for the population without coughing agents. The population of agents was set to 100 and configuration of obstacles from Fig~\ref{fig:gui-world-3}. The colour scale is the same as in the Figs~\ref{fig:sick_increase_large_s_pop100} and \ref{fig:sick_increase_large_s_pop50}. One should note that even for the significant portion of the infected agents in the population, the spread of the disease is limited.}
\label{fig:sick_increase_small_s_pop100_world-3}
\end{figure}






\appendix

\section{Simulation parameters}

\begin{table}
\centering
\begin{tabular}{|c|c|}
\hline
\textbf{Figure} & \textbf{Experiment} \\
\hline
Fig~\ref{fig:sick_increase_large_s_pop100} & exp6  (population 100, world-1)\\
\hline
Fig~\ref{fig:sick_increase_large_s_pop50} & exp6a (population 50, world-1)\\ 
\hline
Fig~\ref{fig:sick_increase_small_s_pop100_world-3} & exp7 (population 100, world-3)  \\
\hline
 &  \\
\hline
\end{tabular}
\caption{Names of the experiments from \texttt{experiements-v2.xml} file for the figures included in the report.}
\end{table}

\end{document}